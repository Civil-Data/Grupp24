% Author: Joel Scarinius Stävmo, Oskar Sundberg, Linus Savinainen, Samuel Wallander Leyonberg  and Gustav Pråmell
% Update: October 1, 2024
% Version: 1.0.0
% License: Apache 2.0

\textbf{Defining a lower bound}:

$ D_n $ is the distance needed, $ D_t $ is the distance traveled, $ N $ is the number of people in the building and $ F $ is the number of floors, using zero index.

\textbf{Assumption I}:

The distance between any floor $ J $ and $ J + 1 $, or  $ J $ and $ J - 1 $ is $ 1 $ and the $ time $ required is $ 1 $.

\textbf{Assumption II}:

Following \textit{I}, a person traveling from e.g., floor $ J $ to floor $ J + 3 $, $ D_n = |J - ( J + 3)|$.

\textbf{Assumption III}:

The elevator has a capacity of $ E $ people.

\textbf{Assumption IV}:

For $ X $ people that start on the same floor and needs to go to the same floor and $ E \geq X $ the lowest $ time $ spent waiting is equal to $ D_n \cdot X $.

\textbf{Proof}:

To prove \textit{IV}, the trivial case is a building with $ N = 0 $. There is no need to move the elevator, resulting in $ 0 \cdot N = 0 $ and $ time = D_n \cdot 0 = 0 $.

\textbf{Simple case}:

A building with $ F = 2 $ and $ N = 1 $, where the person needs to move for floor zero to floor one. It is trivial to see that the best solution is for the elevator to take one person from floor 0 to floor one. Resulting in $ D_n = time = 1 = D_t \cdot 1 $.

\textbf{A more general case}:

A building with $ F = 2 $ and $ P $ people that need to move from floor zero to floor one. The optimal solution is $ P \cdot D_n = P \cdot D_t = time = P $. This is the lower bound of the problem. By proving \textit{IV}, it can be shown that the lower bound is $ D_n \cdot P $.

\textbf{The upper bound}:

To prove the upper bound the following solution can be imagined; in an arbitrary building, move the elevator continuously between any two or more floors that does not have any people. Resulting in $ 1 \cdot \infty = \infty $.
