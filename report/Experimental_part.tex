% Author: Joel Scarinius Stävmo, Oskar Sundberg, Linus Savinainen, Samuel Wallander Leyonberg and Gustav Pråmell
% Update: October 1, 2024
% Version: 1.0.0
% License: Apache 2.0

This section describes the setup of experiments:

Each experiment consists of two JSON files; one representing the building which specifies where all people are located, and another containing the population with a set of initial genomes.
These experiments are saved to files rather than being generated randomly each time to ensure the ability to run the same experiment multiple times for consistency.

One specific building which already had result data available was used to compare this algorithm's results against previously known results.

To thoroughly evaluate the algorithm it was decided to test it on buildings of different sizes (small, medium, large) and with corresponding population sizes.
Given the number of experiments required to be run, buildings with 10, 50, and 100 floors containing populations of 10, 50, and 100 people respectively was selected. Additionally, the specific building mentioned earlier which has 21 floors and 10 people was included.

Buildings were generated with constraints ensuring that each floor could hold between 0 and half of the maximum population.
While it was theoretically possible for all people to be on just two floors in practice the distribution was more balanced. For example, in the small 10-floor building with 10 people the distribution was as follows:

\begin{center}
	\begin{tabular}{c c c}
		\textbf{From} &       & \textbf{To} \\
		\hline
		0             & $\to$ & 6           \\
		0             & $\to$ & 8           \\
		3             & $\to$ & 5           \\
		4             & $\to$ & 7           \\
		5             & $\to$ & 7           \\
		6             & $\to$ & 2           \\
		7             & $\to$ & 0           \\
		8             & $\to$ & 4           \\
		9             & $\to$ & 0           \\
		9             & $\to$ & 2           \\
	\end{tabular}
\end{center}

After generating and validating all buildings and populations, the experiments were run. Each experiment was conducted on a computer with an AMD Ryzen 7 7800X3D and took roughly ten seconds to a minute. In total 150 experiments were run.
\newpage
All experiment used the following settings 

\begin{itemize}
    \item \textbf{Generations:} 1000
    \item \textbf{Population size:} 100
    \item \textbf{Crossover methods:} Heuristic single, Heuristic sequence, Swap crossover
    \item \textbf{Mutation methods:} Increase, Decrease, Swap, Swap and Increase, Swap and Decrease
    \item \textbf{Elitism rate:} 0.1
    \item \textbf{Crossover rate:} 1.0
    \item \textbf{Mutation rate:} 0.1, 0.6
	\item \textbf{Elevator capacity:} 8
\end{itemize}


Due to time constraints the only parameter that was adjusted during the experiments was the mutation rate to limit the number of results to analyze.
To ensure the reliability of results each combination of building and population was tested five times allowing to identify potential anomalies. Experiments were conducted with two mutation rates, 10\% and 60\%, to observe differences in algorithm performance.
The thought process behind the very high mutation rate was to semi-simulate the effects of a longer generation limit without significantly increasing runtime.

The experimental procedure is outlined as follows:

\begin{itemize}
	\item Generate the building and population, or load a pre-existing experiment.
	\item Run the experiment generating a CSV file with results and a PNG with a graph.
	\item Re-run the same experiment with different parameters.
	\item Analyze the results to evaluate the algorithm's performance.
\end{itemize}
