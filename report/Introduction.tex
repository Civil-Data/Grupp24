\textcolor{red}{The recommended software for typesetting assignment reports is \LaTeX. It will allow you to prepare high-quality documents, especially in the area of Computer Science. This document can serve as a template for reports. Each section begins with brief instructions in red text. All the instructions in red, as well as the dummy text, should be removed in the final version to submit. The \LaTeX\ source of this file includes examples of using the most needed commands and environments. You can find plenty of other examples with explanations in many web forums and discussion groups on the Internet. The easiest way to edit your report is to use \url{https://www.overleaf.com/}. Overleaf does not require any setup on your computer, and it is free to create an account.}

\textcolor{red}{The book \textit{Writing for Computer Science} \cite{zobel2014writing} is a useful assistance on how to write properly and present your work when it comes to Computer Science topics. It is a strong recommendation to follow its guidelines and limit the usage of AI tools to generate text. Keep in mind that the examiner is an expert in Evolutionary Computation and therefore, any false information generated by an AI tool is easily notable. Such case may lead to failing the assignment.}

\textcolor{red}{The introduction should briefly introduce the assignment and its purpose.}

This report addresses an investigation for optimizing an elevators route to improve the efficiency of picking up and delivering passengers to their desired floors in the least amount average waiting time.
In large buildings with a lot of floors, efficient elevators are crucial for managing traffic and must serve every one in a reasonable time.
Poorly designed elevator systems will lead to long waiting time, unnecessary stops and unsatisfied users.
Elevator technology have made progress during the years, but many elevators still struggle with finding an efficient way
to serve passengers.

The hypotheses that evolutionary algorithms will be able to find near-optimal or optimal routes for elevators
aiming to maximize the number of passengers served while minimizing travel distance and/or travel time is
the core focus in the report. Different strategies and hyperparameter are experimented with to enable
demonstration of how evolutionary algorithms can be used in such a problem. 

This paper will focus on a statically defined problem, i.e., there will not be more passengers coming to the elevator as time goes on, unlike a normal elevator. 
