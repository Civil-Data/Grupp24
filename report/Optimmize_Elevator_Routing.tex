\textcolor{red}{The second section should present the problem you tackle using your evolutionary approach. Overall, this section should include:}

{\color{red}
	\begin{itemize}
		\item The mathematical formulation considered in your study. Some problems have a clear mathematical model (e.g., Travelling Salesman Problem), while others do not (e.g., $n$-Queens). Based on the problem you chose, search the literature and find a proper way to present the problem.
		\item One paragraph that briefly presents at least 3 published academic works where any evolutionary approach is used to solve the problem. It would be wise to cite here works that influenced your algorithm. This practice saves you time from looking for additional academic resources. You can find more information about reading and searching in the literature in \cite{zobel2014reading}.
		\item The motivation behind the evolutionary approach you decided to develop. A good practice would be to align the motivation with some literature gap found in the academic works you presented above. However, this is not mandatory. You can motivate your selection on the characteristics of the algorithm making it proper for the problem.
	\end{itemize}
}

\textcolor{red}{\textbf{Note:} Change the section's title to match the name of the problem you chose for your assignment.}

\subsection{Mathematical formulation}
\subsection{Similar published academic work}
\subsection{Why this evolutionary approach}