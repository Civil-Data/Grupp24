\textcolor{red}{The second section should present the problem you tackle using your evolutionary approach. Overall, this section should include:}

{\color{red}
	\begin{itemize}
		\item The mathematical formulation considered in your study. Some problems have a clear mathematical model (e.g., Travelling Salesman Problem), while others do not (e.g., $n$-Queens). Based on the problem you chose, search the literature and find a proper way to present the problem.
		\item One paragraph that briefly presents at least 3 published academic works where any evolutionary approach is used to solve the problem. It would be wise to cite here works that influenced your algorithm. This practice saves you time from looking for additional academic resources. You can find more information about reading and searching in the literature in \cite{zobel2014reading}.
		\item The motivation behind the evolutionary approach you decided to develop. A good practice would be to align the motivation with some literature gap found in the academic works you presented above. However, this is not mandatory. You can motivate your selection on the characteristics of the algorithm making it proper for the problem.
	\end{itemize}
}

\textcolor{red}{\textbf{Note:} Change the section's title to match the name of the problem you chose for your assignment.}

\subsection{Mathematical formulation}
\subsection{Similar published academic work}
2.2 
A Genetic Algorithm Based Elevator Dispatching Method For Waiting Time Optimization \cite{tartan2016genetic} explains how they solve a similar problem but with multiple elevators (what they call cars) instead of our one elevator per building. This paper has been our starting point in terms of algorithms and our approaches by following their flow chart provided in the paper.  
 
The paper describes a \%21 improvement with their methods compared to Ghareib paper \cite{gharieb2005optimal} on the same problem. As this problem is not interlay equal to ours, comparison with this would not give a representee outcome. However, by following the approach in the paper we can run our own simulations for our problem space and compare the results to results given with (theoretically) more suited genetic operations.

INVESTIGATION OF OPTIMIZATION TECHNIQUES ON THE ELEVATOR DISPATCHING PROBLEM \cite{ahmed2022investigation} investigates for different approaches in terms of algorithm to the problem at hand.  One of there algorithm to solve the elevator dispatching problem is a genetic algorithm where they use Davis-order crossover, swap mutation and calculate average time for occupants  witch is used as fitness function. There building (problem space) and predefined settings where as following: 
  
The results from their test show that GA where the best performing algorithm for this problem with the definitions above. 
 
This paper presents results that we later compare to ours with our first implementation as given of the flow chart\cite{tartan2016genetic} as well as later implementation with updated genetic operations. Given their result for testing different algorithms on the problem, a genetic algorithm approach seems to contribute with the best result to the elevator dispatching problem.
Crossover and Mutation Operators of Genetic Algorithms \cite{ahmed2022investigation} lists and shortly describes different types of crossover and mutation operations. They describe Wright's heuristic crossover and categorized it as one of the crossover functions how’s offspring are “in the exploration region near the parents” \cite{ahmed2022investigation}. This reduce premature offspring by overcoming local maximum. On further investigation into different implementations of heuristic crossover, the concept seamed as a good crossover function for our problem due to its simplicity and ability to be largely modified to suit our problem and fitness function.



Approach
Our starting point was highly inspired from the flowchart above[1]. Our idea was to implement this as a first approach to have a working code base and to be able to easily switch between genetic operations for experimental reasons. From the papers mentioned in 2.2 Similar papers, a conclusion was drawn that a heuristic crossover would suit our problem for it could be implemented to swap larger sections of genes from different parents. This would be important due to the fact that it is certain sequences of genes that will contribute towards a better result rather than single genes. Our fitness function was at first implementation only focused on giving points for delivering a passenger to its desired floor. This was to make sure that every passenger would arrive. However, we later changed this to have our fitness function take the time spent for the passengers instead together with a penalty time for not delivering a passenger to its destination. Our mutation function plays a fundamental role and is quite aggressive. It swaps genes and can increase or decrease the size of the chromosome. We choose this for avoiding local maximums found by our crossover function. As our implementation of a heuristic crossover function takes larger sequences out of one parent, our thought was that this would not give an explorational function, thus bringing the need for a more aggressive mutation function to compensate for this.


\cite{anily1992swapping}, "Each vertex of a graph initially may contain an object of a known type. A final state, specifying the type of object desired at each vertex, is also given. A single vehicle of unit capacity is available for shipping objects among the vertices. The swapping problem is to compute a shortest route such that a vehicle can accomplish the rearrangement of the objects while following this route."
\section*{Mapping the elevator problem to the swapping problem.}
\begin{itemize}
	\item Vertices are represented as floors in the elevator.
	\item The object is defined as a person.
	\item Initial state consists of N persons waiting at the floors.
	\item The final state is reached when all persons are at their destination floor.
	\item The elevator is a vehicle with a max capacity that moves the persons between floors, i.e. a single vehicle shipping objects among the vertices.
	\item The elevator problem is aiming to reduce the waiting time of the persons in the elevator and thereby reduce the distance traveled.
\end{itemize}


 %---Oskar---%


\subsection{Why this evolutionary approach}
