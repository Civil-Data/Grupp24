\textcolor{red}{The second section should present the problem you tackle using your evolutionary approach. Overall, this section should include:}

{\color{red}
	\begin{itemize}
		\item The mathematical formulation considered in your study. Some problems have a clear mathematical model (e.g., Travelling Salesman Problem), while others do not (e.g., $n$-Queens). Based on the problem you chose, search the literature and find a proper way to present the problem.
		\item One paragraph that briefly presents at least 3 published academic works where any evolutionary approach is used to solve the problem. It would be wise to cite here works that influenced your algorithm. This practice saves you time from looking for additional academic resources. You can find more information about reading and searching in the literature in \cite{zobel2014reading}.
		\item The motivation behind the evolutionary approach you decided to develop. A good practice would be to align the motivation with some literature gap found in the academic works you presented above. However, this is not mandatory. You can motivate your selection on the characteristics of the algorithm making it proper for the problem.
	\end{itemize}
}

\textcolor{red}{\textbf{Note:} Change the section's title to match the name of the problem you chose for your assignment.}

\subsection{Mathematical formulation}
% Author: Joel Scarinius Stävmo, Oskar Sundberg, Linus Savinainen, Samuel Wallander Leyonberg  and Gustav Pråmell
% Update: October 1, 2024
% Version: 1.0.0
% License: Apache 2.0

\textbf{Diffing a lower bound}:

Dn = distance needed, Dt = distance travled, N is the number on persons in the building and F = number off floors.(using zero index)

\textbf{Assumption I}: The distance between any floor J and J+1 or J and J-1 is one and the time required is one.

\textbf{Assumption II}: Following I, a person need to go from floor J to floor J+3, the Dn is equal to |J-(J+3)|.

\textbf{Assumption III}: The elevator has the capacity of N.

\textbf{Assumption IV}: For N persons that starts on the same floor and needs to go to the same floor, the lowest time spent waiting is equal to Dn*N.

\textbf{Proof}:
To prove Assumption IV:
Trivial case: A building with F = X and N = 0, there is no need to move the elevator. Resulting in 0*N=0, the Dn*0 = time = 0.

\textbf{Simple case}: A building with F = 2 and N = 1, where the person needs to move for floor zero to floor one. It is trivial to see that the best solution is for the elevator to take one person from floor 0 to floor one. Resulting in Dn = time = 1 = Dt*1.

\textbf{A more general case}:
A building with 2 floor and N person that need to move from floor zero to floor one. The optimal solution is, N*Dn = N*Dt = Time = N.

This is the lower bound of our problem, by proving IV, it can be shown that the lower bound is Dn*N.

\textbf{The upper bound}}:
To prove the upper bound the following solution can be imagined.

A building with F = 3, N = 1, the person is on floor 0 and like to go to floor 2. The elevator picks up the person but never goes to floor 2, instead the elevator moves between floor zero and floor one for ever. Resulting is  1 * \infty  =   \infty.

\subsection{Similar published academic work}

\cite{anily1992swapping}, "Each vertex of a graph initially may contain an object of a known type. A final state, specifying the type of object desired at each vertex, is also given. A single vehicle of unit capacity is available for shipping objects among the vertices. The swapping problem is to compute a shortest route such that a vehicle can accomplish the rearrangement of the objects while following this route."
\section*{Mapping the elevator problem to the swapping problem.}
\begin{itemize}
	\item Vertices are represented as floors in the elevator.
	\item The object is defined as a person.
	\item Initial state consists of N persons waiting at the floors.
	\item The final state is reached when all persons are at their destination floor.
	\item The elevator is a vehicle with a max capacity that moves the persons between floors, i.e. a single vehicle shipping objects among the vertices.
	\item The elevator problem is aiming to reduce the waiting time of the persons in the elevator and thereby reduce the distance traveled.
\end{itemize}


 %---Oskar---%
\subsection{Why this evolutionary approach}
