\textcolor{red}{The second section should present the problem you tackle using your evolutionary approach. Overall, this section should include:}

{\color{red}
	\begin{itemize}
		\item The mathematical formulation considered in your study. Some problems have a clear mathematical model (e.g., Travelling Salesman Problem), while others do not (e.g., $n$-Queens). Based on the problem you chose, search the literature and find a proper way to present the problem.
		\item One paragraph that briefly presents at least 3 published academic works where any evolutionary approach is used to solve the problem. It would be wise to cite here works that influenced your algorithm. This practice saves you time from looking for additional academic resources. You can find more information about reading and searching in the literature in \cite{zobel2014reading}.
		\item The motivation behind the evolutionary approach you decided to develop. A good practice would be to align the motivation with some literature gap found in the academic works you presented above. However, this is not mandatory. You can motivate your selection on the characteristics of the algorithm making it proper for the problem.
	\end{itemize}
}

\textcolor{red}{\textbf{Note:} Change the section's title to match the name of the problem you chose for your assignment.}

\subsection{Mathematical formulation}
\subsection{Similar published academic work}




\cite{anily1992swapping}, "Each vertex of a graph initially may contain an object of a known type. A final state, specifying the type of object desired at each vertex, is also given. A single vehicle of unit capacity is available for shipping objects among the vertices. The swapping problem is to compute a shortest route such that a vehicle can accomplish the rearrangement of the objects while following this route."
\section*{Mapping the elevator problem to the swapping problem.}
\begin{itemize}
	\item Vertices are represented as floors in the elevator.
	\item The object is defined as a person.
	\item Initial state consists of N persons waiting at the floors.
	\item The final state is reached when all persons are at their destination floor.
	\item The elevator is a vehicle with a max capacity that moves the persons between floors, i.e. a single vehicle shipping objects among the vertices.
	\item The elevator problem is aiming to reduce the waiting time of the persons in the elevator and thereby reduce the distance traveled.
\end{itemize}


 %---Oskar---%
\subsection{Why this evolutionary approach}
