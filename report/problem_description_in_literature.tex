
\cite{anily1992swapping}, "Each vertex of a graph initially may contain an object of a known type. A final state, specifying the type of object desired at each vertex, is also given. A single vehicle of unit capacity is available for shipping objects among the vertices. The swapping problem is to compute a shortest route such that a vehicle can accomplish the rearrangement of the objects while following this route."
\section*{Mapping the elevator problem to the swapping problem.}
\begin{itemize}
	\item Vertices are represented as floors in the elevator.
	\item The object is defined as a person.
	\item Initial state consists of N persons waiting at the floors.
	\item The final state is reached when all persons are at their destination floor.
	\item The elevator is a vehicle with a max capacity that moves the persons between floors, i.e. a single vehicle shipping objects among the vertices.
	\item The elevator problem is aiming to reduce the waiting time of the persons in the elevator and thereby reduce the distance traveled.
\end{itemize}

According to \cite{anily1992swapping},  "Even the simple swapping problem is NP-hard."


