% Author: Joel Scarinius Stävmo, Oskar Sundberg, Linus Savinainen, Samuel Wallander Leyonberg  and Gustav Pråmell
% Update: October 1, 2024
% Version: 1.0.0
% License: Apache 2.0

The report's starting point was highly inspired from the flowchart above \cite{tartan2016genetic}. The report's idea was to implement this as a first approach to have a working code base and to be able to easily switch between genetic operations for experimental reasons. From the papers mentioned in \ref{sec:2_2}, similar papers, a conclusion was drawn that a heuristic crossover would suit the report's problem because it could be implemented to swap larger sections of genes from different parents. This would be important due to the fact that it is certain sequences of genes that will contribute towards a better fitness rather than single genes. This study's fitness function was at first only focused on giving points for delivering a passenger to its desired floor. This was to make sure that every passenger would arrive. However, later this was changed to have the fitness function take the time spent for the passengers instead together with a penalty time for not delivering a passenger to its destination. The report's mutation function plays a fundamental role and is quite aggressive. It swaps genes and can increase or decrease the size of the chromosome. The decision was made for avoiding local minimums found by the crossover function. As the report's implementation of a heuristic crossover function takes larger sequences out of one parent, the thought was that this would not give an exploration function, thus bringing the need for a more aggressive mutation function to compensate for this.
