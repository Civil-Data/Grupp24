% Author: Joel Scarinius Stävmo, Oskar Sundberg, Linus Savinainen, Samuel Wallander Leyonberg  and Gustav Pråmell
% Update: October 1, 2024
% Version: 1.0.0
% License: Apache 2.0


\subsection{Evolutionary approach}
% Clearly describe the algorithm you developed. You
% should clearly explain the evolutionary operators you used and what modifications
% you did to match the problem. It is extremely important to present also a pseudo code
% of your algorithm.

The genetic algorithm implemented in this study is based on the framework presented in a relatively recent paper \cite{tartan2016flow}, which explored a similar problem. The details of the implemented genetic algorithm can be found in \ref{alg:pseudocode_ga}. A key difference in this paper is the absence of a termination criterion; instead, the algorithm is set to execute for a specified number of generations. The flowchart illustrating the original framework is shown in Figure \ref{fig:flowchart}.

\begin{algorithm}
	\caption{Genetic Algorithm}\label{alg:pseudocode_ga}
	\begin{algorithmic}
		\Require $g \geq 0$
		\State Initialize a new population
		\State $g \gets 0$
		\While{\text{g} $ < $ \text{generation limit}}
		\For{\text{all chromosome in population}}
		\State Execute the chromosome on its people
		\EndFor
		\State Calculate the fitness of the population
		\State Sort the population
		\If{\text{want elitism}} \Comment{Start preparing the next generation}
		\State Bring some of the best chromosome to the new population
		\EndIf
		\While{\text{the size of the new population} $ < $ \text{population size}}
		\State Select two parents from the old population
		\If{the crossover chance is fulfilled}
		\State Breed two children using a crossover operator
		\Else
		\State Select the two parents as the two new children
		\EndIf
		\For{\text{each child}}
		\If{the mutation chance is fulfilled}
		\State Apply a mutation operator
		\EndIf
		\EndFor
		\State Append the two children to the next population
		\EndWhile
		\EndWhile
	\end{algorithmic}
\end{algorithm}

\subsubsection{Selection}
% Explain selection algorithm(s)

The chosen selection algorithm is rank-based selection, and the basic principle behind it is to choose two parents based on their ranks rather than their raw fitness values. For example, if the population consists of six chromosome, the best chromosome is given rank six, the second best rank five and so on. The worst chromosome is given rank one. Each chromosome is then given a probability $ P_i $ to be chosen based on their respective ranks, expressed as:

$$ P_i = \frac{\text{rank}_i}{\sum \text{ranks}} $$

That is, the best chromosome would have a probability of $ P_6 = \frac{6}{21} $ to be chosen, the second best $ P_5 = \frac{5}{21} $, and the worst chromosome $ P_1 = \frac{1}{21} $.

\subsubsection{Crossover}
% Explain crossover algorithm(s)
\begin{par}
	The first basic crossover operator is one that simply swaps the last halves of the parents, while respecting the possibility that the two parents may be of different length. For example, if the two parents are
	\[
		[1, 2, 3, 4], [5, 6, 7, 8, 9, 10]
	\]
	the resulting children would be
	\[
		[1, 2, 8, 9, 10], [5, 6, 7, 1, 2]
	\]
	\label{par:swap last halves}
\end{par}

Two last crossover functions where both heuristic crossover functions, meaning the parents fitness score determines how much of the genes from each parent is transferred to its children with favor for the “dominant” parent.

\begin{par}
	The first heuristic crossover function works simply by summarizing both parents' fitness scores and then calculating the parents' score differences in percentage. Random single genes are then transferred to the children where the previously calculated percentage gives a higher chance for transferring its genes based on that percentage.
	\label{par:swap single genes}
\end{par}

\begin{par}
	The second crossover function works in a similar way but with the implementation of a crossover point instead of choosing single genes. This crossover point moves further down the dominant parents chromosome based on the difference between the parents fitness score. With the crossover point instead of choosing single genes, the children get a larger sequence from the dominant parent and therefore more alike the dominant parent.
	\label{par:swap sequnce of genes}
\end{par}

\subsubsection{Mutation}
% Explain mutation algorithm(s)
\label{subsubsec:mutation}

Three basic mutation operators were implemented. Firstly, a swap mutation where two randomly chosen floors are swapped. Secondly, an operator which increases the length of the chromosome by appending a random floor at the end. Lastly, an operator that decreases the length by removing a randomly chosen index of the chromosome. With these three operators, an additional two operators were constructed by combining the effects of swap and either length operator.

% \subsubsection{Replacement} % Elitism
% We never used elitism in the experiments?

\subsection{Solution representation}
% Clearly describe the solution representation you used.
% You can use figures to improve the comprehensibility of this part.

Given the nature of this problem, there is not a typical direct solution representation approach as seen in e.g. the knapsack problem, where the solution representation is also directly what is being manipulated in the genetic algorithm. In this paper's elevator problem, the chromosome is a sequence of floors in which the elevator travels to with the goal to deliver as many people as possible. This situation forced a 'two-way' representation, where the fitness of a chromosome is reflected upon a list of people. Thus, it was decided to encapsulate the list of floors and the list of people under a single object 'Chromosome' as the solution representation.

\subsection{Fitness function}
\label{subsec:fitness_function}
% It is also very important to mention the fitness function you
% used. In many cases, the objective function of the problem is not the same as
% the fitness function used in an evolutionary algorithm

After some basic trial and error, and more importantly, a healthy discussion with the assignment teacher, it was decided to exclusively punish unwanted behavior of the chromosome instead of giving rewards or mixing rewards with punishments. As the ultimate goal of the algorithm is to deliver all passengers in a short average time, a punishment-only approach acts like a 'catch-all' sink. Instead of trying to find and giving rewards for all the different kinds of positive behaviors, it is instead easier to punish any and all negative behaviors. For example, if a passenger wants to travel from floor five to one, that requires an absolute minimum of four floors worth of travel. The implemented fitness function would in that case punish based on the difference in distance needed and the distance that was actually traveled by the passenger. Another simple example would be to punish the chromosome for everyone that did not arrive at their destinations, rather than rewarding for those that did arrive.  This fitness function also takes into account the total time taken for each individual person, starting when the elevator initially starts moving to when the passanger steps out of the elevator. That is, the total time for a person is time spent waiting for the elevator plus the time spent inside the elevator.

\subsection{More code}
\label{subsec:code}
All code for the report is on Github. 
\url{https://github.com/Civil-Data/Grupp24/}
