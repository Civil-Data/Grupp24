Approach
Our starting point was highly inspired from the flowchart above\cite{tartan2016genetic}. Our idea was to implement this as a first approach to have a working code base and to be able to easily switch between genetic operations for experimental reasons. From the papers mentioned in 2.2 Similar papers, a conclusion was drawn that a heuristic crossover would suit our problem for it could be implemented to swap larger sections of genes from different parents. This would be important due to the fact that it is certain sequences of genes that will contribute towards a better result rather than single genes. Our fitness function was at first implementation only focused on giving points for delivering a passenger to its desired floor. This was to make sure that every passenger would arrive. However, we later changed this to have our fitness function take the time spent for the passengers instead together with a penalty time for not delivering a passenger to its destination. Our mutation function plays a fundamental role and is quite aggressive. It swaps genes and can increase or decrease the size of the chromosome. We choose this for avoiding local maximums found by our crossover function. As our implementation of a heuristic crossover function takes larger sequences out of one parent, our thought was that this would not give an explorational function, thus bringing the need for a more aggressive mutation function to compensate for this.
