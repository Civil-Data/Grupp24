% Author: Joel Scarinius Stävmo, Oskar Sundberg, Linus Savinainen, Samuel Wallander Leyonberg  and Gustav Pråmell
% Update: October 1, 2024
% Version: 1.0.0
% License: Apache 2.0

The paper mainly looked at different crossover functions impact on our genetic algorithm ability to find solutions for various buildings configurations. For the statistics the conclusions drawn are:

\begin{center}
	\begin{tabular}{|c|c|}
		\hline
		Building & Crossover function  \\ \hline
		1        & Vilken som var bäst \\ \hline
		2        & Vilken som var bäst \\ \hline
		3        & Vilken som var bäst \\ \hline
	\end{tabular}
\end{center}
Because the algorithm is not forced to find a solution where all people reach their destination floor, our implementations include a time penalty discussed in (3.x), this to incentivize the algorithm to serve all people to their destination floor. Further research on the impact of this time penalty in relation to the number of people in the building and their configurations and configurations.

A finding from the paper is that a higher mutation rate seems to be better. To determine this further research in this area is needed. Is a higher mutation rate better? If so, what is the optimal range?

In addition, this report does not focus on the selection algorithm for the problem. Further research on which selection algorithm will perform best for the problem at hand would be of interest. Also, it could be interesting to investigate how other mutation functions could impact the results.