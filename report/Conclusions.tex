% Author: Joel Scarinius Stävmo, Oskar Sundberg, Linus Savinainen, Samuel Wallander Leyonberg  and Gustav Pråmell
% Update: October 1, 2024
% Version: 1.0.0
% License: Apache 2.0

The paper mainly looked at different crossover functions impact on our genetic algorithm ability to find solutions for various buildings configurations. For the statistics the conclusions drawn are:

We started of with a fitness function that gave 10 points for each person served, which in the end promoted the algorithm to serve all people in the building. The problem with this was that it tended to give long genomes because it was more beneficial to serve all people than to serve them quickly. We then changed fitness function to one that gave huge penalties for all people that wasn't served by the elevator when the route was finished. It also increased the score with one for each person not arriving to its destination for each floor traveled. This means that the lower score the better. It also made it possible for us to analyze the average waiting time and this was also a benchmark that most papers we found used for this kind of problem.

We didn't investigate how different mutation functions would affect the results. We used 5 different ones that was randomly selected when mutation occurred. The mutation functions we used was swapping a random element in a genome, increase the length of a genome, decrease the length of a genome and the combinations of swapping and increasing or decreasing the length of a genome. These are all thoroughly explained in (3. M).

An approach that would have been interesting to test is to implement elements of heuristic operators into Davis-order crossover. This would allow for more segments being made but also have the more “dominant” parent transfer more of its genes into the next generations.

\begin{center}
	\begin{tabular}{|c|c|}
		\hline
		Building & Crossover function  \\ \hline
		1        & Vilken som var bäst \\ \hline
		2        & Vilken som var bäst \\ \hline
		3        & Vilken som var bäst \\ \hline
	\end{tabular}
\end{center}
Because the algorithm is not forced to find a solution where all people reach their destination floor, our implementations include a time penalty discussed in (3.x), this to incentivize the algorithm to serve all people to their destination floor. Further research on the impact of this time penalty in relation to the number of people in the building and their configurations and configurations.

A finding from the paper is that a higher mutation rate seems to be better. To determine this further research in this area is needed. Is a higher mutation rate better? If so, what is the optimal range?

In addition, this report does not focus on the selection algorithm for the problem. Further research on which selection algorithm will perform best for the problem at hand would be of interest. Also, it could be interesting to investigate how other mutation functions could impact the results.